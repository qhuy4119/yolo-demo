\documentclass[20pt, a4paper]{article}
\usepackage[urlcolor=blue, colorlinks, linkcolor=blue]{hyperref}


\begin{document}
\title{Báo cáo đồ án cuối kỳ \\
Máy học thống kê\\
Yolo Object Detection}
\author{1712495 - Nguyễn Quang Huy}
\maketitle

\renewcommand{\abstractname}{Tóm tắt}
\begin{abstract}
	Xây dựng website cho phép người dùng sử dụng mô hình Yolov4-Tiny để nhận dạng đối tượng thường ngày 
	hoặc mô hình đã được điều chỉnh để nhận dạng 5 loài động vật: báo(cheetah), tinh tinh(chimpanzee), sư tử(lion), hươu vàng(hog deer), gấu chó(sun bear). 
\end{abstract}

\renewcommand{\contentsname}{Mục lục}
\tableofcontents

\section{Mô hình Yolov4-Tiny}

\href{https://github.com/AlexeyAB/darknet#pre-trained-models}{Yolov4-Tiny} 
là mô hình nhận dạng đối tượng được huấn luyện trên tập dữ liệu MS COCO với kết quả 40.2\% mAP@0.5.

Để sử dụng mô hình, ta build framework \href{https://github.com/AlexeyAB/darknet}{darknet} và dùng nó cùng với trọng số của mô hình đã được huấn luyện để dự đoán đối tượng trong ảnh bất kỳ

\section{Mô hình nhận dạng động vật}
\subsection{Tổng quan}
Mô hình này là Yolov4-Tiny được huấn luyện tiếp để nhận dạng 5 loài động vật: báo, tinh tinh, sư tử, hươu vàng, gấu chó.
\subsection{Tập dữ liệu}
\begin{itemize}
	\item
Tập dữ liệu gồm 2093 hình ảnh của 5 loài động vật, được tải về từ chức năng tìm kiếm hình ảnh của Google.\\
Tham khảo các bước cụ thể trong quy trình thu thập dữ liệu tại đây: \href{https://www.pyimagesearch.com/2017/12/04/how-to-create-a-deep-learning-dataset-using-google-images/}{How to create a deep learning dataset using Google Images}

\begin{tabular}{l | r}
	\textbf{Loài} & \textbf{Số hình ảnh} \\
	\hline
	Báo & 538 \\
	Tinh tinh & 561 \\
	Sư tử & 338 \\
	Hươu vàng & 377 \\
	Gấu chó & 279 \\
\end{tabular}

\item
Sau đó dữ liệu được dán nhãn thủ công theo format của Yolo

\item
Cuối cùng, dữ liệu được chia thành 3 tập train, validation, test bằng 
sklearn.model\_selection.train\_test\_split với tỉ lệ lần lượt 75\%, 15\%, 10\%
\end{itemize}

\subsection{Huấn luyện}
\subsubsection{Cách thực hiện}
Để huấn luyện tiếp(fine tuning) mô hình Yolov4-Tiny có sẵn, ta làm theo 
\href{https://github.com/AlexeyAB/darknet#how-to-train-tiny-yolo-to-detect-your-custom-objects}{hướng dẫn của framework darknet}

Cụ thể:
\begin{itemize}
	\item Sử dụng Google Colab
	\item Chọn batch và subdivisions phù hợp với cấu hình máy tính
	\item Chọn max\_batch tùy theo số loại đối tượng(class). Đây chính là số epoch. \\
		Darknet khuyến cáo max\_batch = max(classes*2000, 6000, number of training images)
	\item Chọn network size: width và height. Các ảnh input sẽ được resize về kích thước này
\end{itemize}

\subsubsection{Kết quả}
\begin{enumerate}
	\item Huấn luyện 2 mô hình có network size khác nhau: (width=544, height=544) và (width=640, 640). Số max\_batch đều là 10000 theo công thức của darknet
	\item Sau 10000 max\_batch, kết quả của cả 2 mô hình trên tập validation vào khoảng 50\% mAP@0.5. 
	\item Thử tiếp tục huấn luận thêm 10000 max\_batch nữa thì đạt được kết quả: <placeholder>
\end{enumerate}

\end{document}

